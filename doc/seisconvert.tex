\documentclass[DIV=10, parskip=half]{scrartcl}
\usepackage[fleqn]{amsmath}
\usepackage{mathspec}
\usepackage{xltxtra}
\usepackage{amssymb}
\usepackage{enumerate}
\usepackage{tikz}
\usepackage{ragged2e}
\usepackage{polyglossia}
\usepackage{microtype}
\setdefaultlanguage{english}

\addtokomafont{disposition}{\rmfamily}
\newcommand{\R}{\mathbb{R}}
\newcommand{\C}{\mathbb{C}}
\newcommand{\N}{\mathbb{N}}
\newcommand{\qed}{\hfill\(\Box\)\hspace{-10pt}\strut}

\makeatletter
\let\TagsLeftOn\tagsleft@true
\let\TagsLeftOff\tagsleft@false
\makeatother

\newcommand{\x}[1]{\textbackslash{}x#1}

\DeclareTextFontCommand{\textref}{\rmfamily}

\renewenvironment{quote}
{\list{}{
  \setlength{\rightmargin}{0cm}
  \setlength{\leftmargin}{0.75cm}}%
\item\relax\ignorespaces}
{\unskip\unskip\endlist}

\newenvironment{openingquote}
{\list{}{
  \setlength{\rightmargin}{0cm}
  \setlength{\leftmargin}{5cm}}%
\item\relax\fontsize{8pt}{12pt}\selectfont\ignorespaces}
{\unskip\unskip\endlist}

\newenvironment{literature}
{\list{}{
  \setlength{\leftmargin}{10pt}
  \setlength{\labelwidth}{0pt}
  \setlength{\labelsep}{0pt}
  \setlength{\itemindent}{-10pt}
  \setlength{\itemsep}{-2pt}
}\RaggedRight}
{\endlist}

\newcommand\litref[1]{\textref{[#1]}}
\newenvironment{refliterature}
{\list{}{
  \setlength{\leftmargin}{45pt}
  \setlength{\labelwidth}{45pt}
  \setlength{\labelsep}{0pt}
  \setlength{\itemindent}{0pt}
  \setlength{\itemsep}{-2pt}
  \let\makelabel\litref
}\RaggedRight}
{\endlist}

%\sloppy

\newcommand{\docline}[1]{%
  \begin{quote}
  \raggedright\texttt{#1}
  \end{quote}
}

\sloppy

\begin{document}

\strut

\vspace{1cm}

\centerline{\huge \textbf{Seisconvert Documentation}}

\vspace{1cm}

\section{Introduction}

The Seisconvert library is a library for handling of several seismic data formats.
It supports KUM-Y, MiniSEED, SEG-Y and WAV.

\section{Components}

\subsection{MiniSEED}

The API for MiniSEED is based on the header file “miniseed\_file.h”.
All the functions of this module handle with the type “miniseed\_file\_t”, which represents an open MiniSEED file.

\subsubsection{Reading}

A file can be opend with
\docline{miniseed\_file\_open(const char *filename)}
This will return a miniseed\_file\_t ready for reading or NULL on error.

Now one can repeatedly call either
\docline{miniseed\_file\_read\_int\_frame(miniseed\_file\_t *file, int32\_t *frame)}
or
\docline{miniseed\_file\_read\_double\_frame(miniseed\_file\_t *file, double *frame)}
until a negative value is returned.
This will read one sample from the file in each call and stores it to the frame pointer.
The int variant of this function stores a 32 bit integer, the double variant stores a normalised double, i.\,e.\ 1.0 as double is the same as 0x7f\,ff\,ff\,ff as integer.

The read functions return \(0\) if a sample was read, \(-1\) if there was no more data and \(-3\) in case of an error.

When you are done with reading, the file must be closed with
\docline{miniseed\_file\_close(miniseed\_file\_t *file)}

\subsubsection{Writing}

If the file should instead be opened for writing, one may call
\docline{miniseed\_file\_create(const char *filename)}
This will return a file in write mode or NULL in case of an error.

If a file has been opened in write mode, the sample rate must be set before any data can be written.
This is done by calling
\docline{miniseed\_file\_set\_sample\_rate(miniseed\_file\_t *file, uint32\_t sample\_rate)}

The start time must be set as well, i.\,e.\ the time, when the recording was started.
This is done with
\docline{miniseed\_file\_set\_start\_time(miniseed\_file\_t *file, struct taia *t)}

After initialisation, one can optionally set the channel metadata with
\docline{miniseed\_file\_set\_info(miniseed\_file\_t *file, const char *station, const char *location, const char *channel, const char *network)}
The fields may be NULL, in which case the specific info remains unchanged.

When all fields have been set, one can start writing samples to the file.
This is done with the functions
\docline{miniseed\_file\_write\_int\_frame(miniseed\_file\_t *file, const int32\_t *frame)}
and
\docline{miniseed\_file\_write\_double\_frame(miniseed\_file\_t *file, const double *frame)}

When you are done with writing, the file must be closed with
\docline{miniseed\_file\_close(miniseed\_file\_t *file)}
This is especially important, as the last block will not be written to disk until this function is called.

\subsection{WAV}

The API for WAV is based on the header file “wav\_file.h”.
All the functions of this module handle with the type “wav\_file\_t”, which represents an open WAV file.

\end{document}
