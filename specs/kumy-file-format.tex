\documentclass[DIV=10]{scrartcl}
\usepackage[fleqn]{amsmath}
\usepackage{mathspec}
\usepackage{xltxtra}
\usepackage{amssymb}
\usepackage{enumerate}
\usepackage{tikz}
\usepackage{polyglossia}
\usepackage{microtype}
\setdefaultlanguage{english}

%\setromanfont{Linux Libertine O}
%\setmathfont(Digits,Latin){Linux Libertine O}

\addtokomafont{disposition}{\rmfamily}
\newcommand{\R}{\mathbb{R}}
\newcommand{\C}{\mathbb{C}}
\newcommand{\N}{\mathbb{N}}
\newcommand{\qed}{\hfill\(\Box\)\hspace{-10pt}\strut}

\makeatletter
\let\TagsLeftOn\tagsleft@true
\let\TagsLeftOff\tagsleft@false
\makeatother

%\pagestyle{myheadings}
%\markright{\textup{The KUM-Y File Format \hfill September 2014}}

\newcommand{\x}[1]{\textbackslash{}x#1}

\begin{document}

\strut

\vspace{1cm}

\centerline{\huge \textbf{The KUM-Y File Format%
\footnote{\raggedright Date of this document: 2014-09-08.
Permanent ID of this document: bd59ae8cc054eee9b8e2c779ba1b6ab9.}}}

\vspace{1cm}

\section{Introduction}

The KUM-Y file format is a format for efficient storing, transmitting and processing of seismic data, generated by various kinds of recorders.

There are already formats in use for this purpose, most notably SEG-Y and SEED, but these formats are subject to several disadvantages, which the format presented in this paper tries to address.

\section{Conventions}

A KUM-Y file can be seen as a finite sequence of 8 bit bytes.
Handling and storage of this byte stream is up to the application and the operating system.
This could be a file on a hard disc or SD card, but a TCP stream would be perfectly acceptable as well.

Characters may be depicted by their hexadecimal representation.
So the string represented by “\x{48}\x{65}\x{6c}\x{6c}\x{6f}\x{20}\x{57}\x{6f}\x{72}\x{6c}\x{64}\x{21}” is the same as “Hello World!”.

\subsection{Byte Order}

All binary integers are in little endian format.
Signed integers shall be interpreted as two’s complement.

\subsection{Strings}

Strings are sequences of 8 bit bytes.
Any characters are acceptable, though for maximum portability the content should be restricted to UTF-8 encoded text.

Strings may be of arbitrary length as they are always embedded with their byte length aside.

\subsection{Times and Durations}

Times and durations are encoded as 128 bit TAIA labels.
A TAIA label consists of an unsigned 64 bit integer \(s\), an unsigned 32 bit integer \(n\) and an unsigned 32 bit integer \(a\).
\(n\) and \(a\) are limited to between 0 and 999,999,999 inclusive.

The TAIA label then encodes the duration
\[
  d = \left(s + n\cdot10^{-9} + a\cdot10^{-18}\right) \cdot 1\text{\,s}.
\]
If the label represents an absolute time, then it is calculated as
\[
  t = t_0 + d - 2^{62}\text{\,s}
\]
where \(t_0\) is the moment that began 1970\,TAI.

\section{Structure of a KUM-Y File}

The general structure of a KUM-Y file is depicted in figure \ref{structure}.

\begin{figure}[ht]
\centerline{\begin{tikzpicture}
\draw
  (0, 0) rectangle +(4, -0.6) +(2, -0.3) node{Magic Number}
  ++(0, -0.6) rectangle +(4, -0.6) +(2, -0.3) node{Text Metadata}
  ++(0, -0.6) rectangle +(4, -0.6) +(2, -0.3) node{Binary Header}
  ++(0, -0.6) rectangle +(4, -0.6) +(2, -0.3) node{Data Frame}
  ++(0, -0.6) rectangle +(4, -0.6) +(2, -0.3) node{\(\dotsm\)}
  ++(0, -0.6) rectangle +(4, -0.6) +(2, -0.3) node{Data Frame};
\end{tikzpicture}}
\caption{Structure of a KUM-Y File}
\label{structure}
\end{figure}

\subsection{Magic Number}

Every KUM-Y file must start with a byte sequence, that identifies the file as KUM-Y format.
The sequence is “\x{4b}\x{55}\x{4d}\x{2d}\x{59}\x{20}\x{31}\x{0a}” (i.\,e.\ “KUM-Y 1” followed by a carriage return) for version 1 of the format.

The magic number may change in future revisions of the format to allow easy detection of the format version.

\subsection{Text Metadata}

After the magic number follows human readable information associated with the file.
This text metadata is a string of arbitrary length, encoded as the byte length of the string in decimal with ascii digits, followed by \x{0a}, followed by the string data, followed by \x{0a}\x{04}.

The length must be encoded in the shortest possible manner, so leading zeros are disallowed except when the length is zero itself.

With this definition the file can be easily opened with any text editor and the textual information can be inspected.

On the other hand, this metadata is of solely informative interest.
It shall not be interpreted in any way by processing applications conforming to this document.

\subsection{Binary Header}

The binary header encodes all the information, that is necessary to read the time series in the file.

\subsection{Data Frames}

The data frames contain the actual time series data, as well as some additional data of interest, such as the estimated clock skew.
Each data frame is protected by a checksum to detect possible transmission errors.

\end{document}
